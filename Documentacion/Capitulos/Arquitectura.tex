La arquitectura del sistema se basa en un modelo híbrido que combina elementos centralizados y distribuidos para optimizar la distribución de contenido de video.

\subsection*{Componentes Principales}

\subsubsection*{Servicio Central (Registry Service)}
El servicio central actúa como un registro y coordinador del sistema, implementado como un microservicio Spring Boot que:

\begin{itemize}
    \item Mantiene un registro de todos los nodos activos en la red
    \item Almacena información sobre qué fragmentos posee cada nodo
    \item Proporciona APIs REST para consultas de ubicación de fragmentos
    \item Gestiona el sistema de notificaciones Pub/Sub con Redis
\end{itemize}

\textbf{Endpoints principales:}
\begin{itemize}
    \item \texttt{POST /api/register} - Registro de nuevos nodos
    \item \texttt{GET /api/nodes} - Lista de nodos registrados
    \item \texttt{GET /api/fragment/\{fragmentId\}} - Localización de fragmentos
    \item \texttt{POST /api/updateFragments} - Actualización de inventario de fragmentos
\end{itemize}

\subsubsection*{Nodos P2P}
Cada nodo P2P es un microservicio independiente que:

\begin{itemize}
    \item Almacena fragmentos de video localmente
    \item Sirve fragmentos a otros nodos mediante HTTP
    \item Se registra automáticamente en el servicio central al iniciar
    \item Escucha notificaciones de nuevos fragmentos disponibles
    \item Implementa lógica de descarga automática de fragmentos faltantes
\end{itemize}

\textbf{Endpoints principales:}
\begin{itemize}
    \item \texttt{GET /fragment/\{id\}} - Descarga de fragmentos
    \item \texttt{POST /fragment/receive} - Recepción de fragmentos
\end{itemize}

\subsubsection*{Sistema Pub/Sub con Redis}
Redis actúa como broker de mensajes para:

\begin{itemize}
    \item Notificar a todos los nodos cuando hay nuevos fragmentos disponibles
    \item Mantener la sincronización del estado del sistema
    \item Reducir la carga en el servicio central mediante notificaciones asíncronas
\end{itemize}

\subsection*{Flujo de Comunicación}

\begin{enumerate}
    \item Los nodos se registran en el servicio central al iniciar
    \item Cuando un nodo recibe un nuevo fragmento, notifica al servicio central
    \item El servicio central publica un evento en Redis
    \item Otros nodos reciben la notificación y pueden solicitar el fragmento
    \item La transferencia de fragmentos ocurre directamente entre nodos (P2P)
\end{enumerate}

\subsection*{Escalabilidad y Tolerancia a Fallos}

El sistema está diseñado para escalar horizontalmente:

\begin{itemize}
    \item Nuevos nodos pueden unirse dinámicamente
    \item La carga se distribuye automáticamente entre nodos
    \item Si un nodo falla, otros nodos pueden tener copias de sus fragmentos
    \item El servicio central puede replicarse para alta disponibilidad
\end{itemize}