\subsection*{Cronograma del Proyecto}

El proyecto se desarrolló en un período intensivo de 5 días (4-8 de agosto de 2025), con una distribución equitativa de responsabilidades entre los tres integrantes del equipo.

\subsection*{Distribución de Actividades por Día}

\subsubsection*{Lunes 4 de Agosto - Planificación y Diseño}

\textbf{Aaron Rodrigo Ramos Reyes:}
\begin{itemize}
    \item Análisis de requisitos del sistema P2P
    \item Diseño de la arquitectura general del sistema
    \item Definición de APIs REST para el servicio central
    \item Creación de diagramas de arquitectura (PlantUML)
\end{itemize}

\textbf{Oscar Martinez Barrales:}
\begin{itemize}
    \item Investigación de tecnologías P2P y microservicios
    \item Diseño de la estructura de datos para fragmentos
    \item Definición de protocolos de comunicación entre nodos
    \item Configuración inicial del entorno de desarrollo
\end{itemize}

\textbf{Oswaldo Mejia Garcia:}
\begin{itemize}
    \item Análisis de patrones de distribución de contenido
    \item Diseño del sistema Pub/Sub con Redis
    \item Configuración de Docker y Docker Compose
\end{itemize}

\subsubsection*{Martes 5 de Agosto - Desarrollo del Servicio Central}

\textbf{Aaron Rodrigo Ramos Reyes:}
\begin{itemize}
    \item Implementación del servicio central (Spring Boot)
    \item Desarrollo de controladores REST
    \item Configuración de Spring Data Redis
    \item Implementación de DTOs (NodeRegistration, FragmentEvent)
\end{itemize}

\textbf{Oscar Martinez Barrales:}
\begin{itemize}
    \item Desarrollo de la lógica de registro de nodos
    \item Implementación del sistema de localización de fragmentos
    \item Configuración de validaciones con Hibernate Validator
    \item Desarrollo de servicios de negocio
\end{itemize}

\textbf{Oswaldo Mejia Garcia:}
\begin{itemize}
    \item Configuración de Redis Pub/Sub
    \item Implementación de listeners de eventos
    \item Desarrollo de la lógica de notificaciones
\end{itemize}

\subsubsection*{Miércoles 6 de Agosto - Desarrollo de Nodos P2P}

\textbf{Aaron Rodrigo Ramos Reyes:}
\begin{itemize}
    \item Implementación de la aplicación P2P Node
    \item Desarrollo de controladores para fragmentos
    \item Implementación de auto-registro en el servicio central

\end{itemize}

\textbf{Oscar Martinez Barrales:}
\begin{itemize}
    \item Desarrollo de NodeClient para comunicación P2P
    \item Implementación de descarga de fragmentos
    \item Configuración de RestTemplate para comunicación HTTP
\end{itemize}

\textbf{Oswaldo Mejia Garcia:}
\begin{itemize}
    \item Implementación de almacenamiento local de fragmentos
    \item Desarrollo de la lógica de sincronización
    \item Configuración de variables de entorno
\end{itemize}

\subsubsection*{Jueves 7 de Agosto - Integración y Testing}

\textbf{Aaron Rodrigo Ramos Reyes:}
\begin{itemize}
    \item Integración completa del sistema
    \item Configuración de Docker Compose
\end{itemize}

\textbf{Oscar Martinez Barrales:}
\begin{itemize}
    \item Testing de APIs con Postman
    \item Validación de transferencia de fragmentos
\end{itemize}

\textbf{Oswaldo Mejia Garcia:}
\begin{itemize}
    \item Validación de notificaciones en tiempo real
    \item Testing de tolerancia a fallos
\end{itemize}

\subsubsection*{Viernes 8 de Agosto - Documentación y Entrega}

\textbf{Aaron Rodrigo Ramos Reyes:}
\begin{itemize}
    \item Redacción de documentación técnica
    \item Creación de instructivos de instalación
    \item Preparación de la presentación final
    \item Revisión final del código
\end{itemize}

\textbf{Oscar Martinez Barrales:}
\begin{itemize}
    \item Documentación de APIs con OpenAPI/Swagger
    \item Creación de guías de uso con Postman
    \item Documentación de arquitectura
    \item Testing final del sistema completo
\end{itemize}

\textbf{Oswaldo Mejia Garcia:}
\begin{itemize}
    \item Documentación de despliegue con Docker
    \item Creación de scripts de automatización
    \item Documentación de configuración
    \item Preparación de entregables finales
\end{itemize}

\subsection*{Metodología de Trabajo}

\begin{itemize}
    \item \textbf{Control de versiones:} Git con ramas por funcionalidad
    \item \textbf{Comunicación:} Discord para coordinación continua
    \item \textbf{Testing continuo:} Validación diaria de integraciones
\end{itemize}

